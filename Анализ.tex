\section{Анализ предметной области}
\subsection{Принципы работы web-платформы для поиска работы}

Web-платформа для поиска работы — это программно-информационная система, предназначенная для автоматизации процессов взаимодействия между соискателями и работодателями. Она позволяет пользователям эффективно находить друг друга, обмениваться информацией и заключать трудовые соглашения. Основная цель платформы — упрощение поиска работы для соискателей и подбора персонала для работодателей, минимизируя временные и ресурсные затраты.

\subsubsection{Регистрация и управление пользователями}

Для использования платформы пользователи проходят процесс регистрации. Соискатели создают профиль, указывая личные данные, а также информацию о профессиональном опыте, образовании и навыках. Работодатели регистрируются как представители компании, указывая данные о компании и контактные данные. После регистрации пользователи получают доступ к личному кабинету, где могут управлять своими данными, резюме или вакансиями. Платформа обеспечивает безопасность данных пользователей, используя шифрование и механизмы авторизации, а также соответствует требованиям законодательства о защите персональных данных.

\subsubsection{Создание и публикация резюме}

Соискатели могут создавать резюме, указывая профессиональные навыки, опыт работы, образование, желаемую зарплату, тип занятости и другие параметры. Платформа предоставляет шаблоны резюме для упрощения процесса заполнения. После создания резюме проходит модерацию, чтобы исключить некорректные или мошеннические записи. Опубликованные резюме становятся доступны для просмотра работодателями, которые могут фильтровать их по заданным критериям.

\subsubsection{Создание и публикация вакансий}

Работодатели могут публиковать вакансии, указывая требования к кандидатам, условия работы, а также описание компании и контактные данные. Как и резюме, вакансии проходят модерацию, чтобы обеспечить их соответствие правилам платформы и законодательству. Опубликованные вакансии становятся доступны для соискателей, которые могут искать их с помощью фильтров.

\subsubsection{Поиск и отклики}

Платформа предоставляет функционал поиска, который позволяет соискателям находить подходящие вакансии, а работодателям — подходящих кандидатов. Соискатели могут использовать фильтры и ключевые слова для поиска вакансий, а затем отправлять отклики, прикрепляя своё резюме. Работодатели, в свою очередь, могут искать резюме по аналогичным критериям и отправлять приглашения на собеседование.

\subsection{Преимущества и недостатки web-платформ для поиска работы}
\subsubsection{Преимущества для соискателей и работодателей}

Удобство: Соискатели могут искать работу в любое время и из любого места с доступом к интернету, экономя время на поездки и встречи. Работодатели могут быстро размещать вакансии и получать отклики без необходимости проводить массовые собеседования.

Широкий охват: Платформа предоставляет доступ к большому количеству вакансий и кандидатов, включая редкие специальности и удалённые регионы, что недоступно в традиционных методах поиска.

Экономия ресурсов: Соискатели экономят на создании и печати резюме, а работодатели — на затратах на рекламу вакансий и первичный отбор кандидатов.

Прозрачность: Соискатели могут видеть отзывы о компаниях, а работодатели — рейтинги кандидатов, что помогает принимать более обоснованные решения.

\subsubsection{Недостатки для соискателей и работодателей}

Качество информации: Соискатели могут столкнуться с устаревшими или некорректными вакансиями, а работодатели — с неактуальными резюме, если модерация недостаточно эффективна.

Конкуренция: Высокая конкуренция среди соискателей и работодателей может затруднять поиск.

Безопасность: Платформа подвержена рискам утечки персональных данных, что требует строгого соблюдения законодательства о защите данных.

Ограничения взаимодействия: Отсутствие личного контакта на этапе отбора может затруднять оценку soft skills соискателя или корпоративной культуры компании.

\subsection{История развития платформ для поиска работы}

Первые платформы для поиска работы начали появляться в конце 1990-х годов, с развитием интернета и переходом традиционных методов (газеты, объявления) в онлайн. Одной из первых известных платформ стала Monster.com, запущенная в 1994 году в США, которая позволяла размещать резюме и вакансии онлайн. В 1997 году появилась CareerBuilder, а в 2003 году — LinkedIn, который добавил социальный аспект, позволяя пользователям строить профессиональные сети.

В России история онлайн-платформ для поиска работы началась в конце 1990-х — начале 2000-х годов. В 1999 году был запущен Job.ru, один из первых российских сервисов, ориентированных на размещение вакансий и резюме. В 2000 году появилась платформа HeadHunter (hh.ru), которая со временем стала лидером рынка в России. В 2005 году запустился SuperJob, предложивший дополнительные функции, такие как тестирование кандидатов.